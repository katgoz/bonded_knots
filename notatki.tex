\documentclass{article}
\usepackage[utf8]{inputenc}
\usepackage[polish]{babel}

\begin{document}

\begin{verbatim}
W katalogu 9 jest podział na grupy z 1 grafem lub kilkoma, wewnątrz tego są foldery: 
-> "8w"-grupę można uprościć do 8w, 
-> "to samo"-albo nie da się nic zrobić albo część da się uprościć do 8w i pozostałe to ten sam 9w, 
-> "+1" - dla danego wielomianu yamady są 2 różne grupy

Yamada polinomial: 18 groups:
-> 8 groups with 1 graph inside:
    -> groups wih 8 vertices: 1
    -1_-1_-1_0_-2_1_-1_-1_-1_-3_-2_-3_-2_0_-2_1

-> 10 groups with multiple graphs:
    -> groups with 8 vertices: 3
    -1_2_0_2_3_2_3_1_1_1_-1_2_0_1_1_1
    1_2_0_3_2_3_5_4_5_3_2_3_0_2_1
    1_1_1_0_2_-1_1_1_1_3_2_3_2_0_2_-1
\end{verbatim}

\end{document}
